\part{Thompson}

\chapter{基本类型及定义}

\section{布尔类型}

布尔类型表示为\hs{Bool},包括两个布尔值\hs{True}和\hs{False}。
这些值来表示某种测试的结果,如测试两个数字是否相等,第一个数字是否小于第二个数字。
布尔类型上的主要操作有逻辑与(\hs{\&\&}),逻辑或(\hs{||})和逻辑非(\hs{not})。
这些操作的语义可以用下面的真值表来表示。

\begin{center}
\begin{tabular}{cc|cc}
  $t_1$ & $t_2$ & $t_1$ \hs{\&\&} $t_2$ & $t_1$ || $t_2$ \\
  \hline
  \hs{True} & \hs{True} & \hs{True} & \hs{True}\\
  \hs{True} & \hs{False} & \hs{False} & \hs{True}\\
  \hs{False} & \hs{True} & \hs{False} & \hs{True}\\
  \hs{False} & \hs{False} & \hs{False} & \hs{False}\\
\end{tabular}

\medskip

\begin{tabular}{c|c}
  $t$  & \hs{not} $t$\\
  \hline
  \hs{True} & \hs{False}\\
  \hs{False} & \hs{True}\\
\end{tabular}

\end{center}

在这些基本操作的基础上,我们可以定义新的逻辑异或操作\hs{exor}:
\begin{code}
exOr :: Bool -> Bool -> Bool
exOr x y = (x || y) && not (x && y)
\end{code}
第一行说明我们要定义一个函数名为\hs{exOr}的操作,而这个操作是接受两个布尔值而返回一个布尔值。
第二行给出了具体的定义,他接受两个参数\hs{x}和\hs{y},利用已定义的操作通过表达式来定义新的操作。
在Haskell中,我们可以省略第一行,但是我们还是鼓励大家写出类型,增加可读性。

\begin{demo}
*Main> exOr True False
True
*Main> exOr True True
False
\end{demo}

在函数定义式,我们可以在定义的左边用布尔值,增加可读性。比如,\hs{exOr}也可以定义如下。
\begin{code}
exOr True y  = not y
exOr False y = y
\end{code}
这种定义方式叫做模式匹配(pattern matching),我们以后还会具体讨论。

对于新定义的函数,我们可以利用Quickcheck来测试函数是否满足某个性质。
\begin{code}
prop_exOr :: Bool -> Bool -> Bool
prop_exOr x y = exOr x y == exOr y x
\end{code}

\begin{exercise}
\hs{x}和\hs{y}的异或为真如果\hs{x}为真或\hs{y}为伪,或\hs{x}为伪或\hs{y}为真。
根据这个说明给出\hs{exOr}的另外一个定义。
\end{exercise}

\begin{exercise}
通过模式匹配,我们可以给予逻辑操作的真值表。完成下面的\hs{xeOr}的真值表的定义。
\begin{xcode}
exOr True True   = ...
exOr True Falseo = ...
exOr False True  = ...
exOr False False = ...
\end{xcode}
\end{exercise}

\section{整数类型:\hs{Integer}和\hs{Int}}

整数类型表示为\hs{Integer},包含所有的整数(正整数,副整数,零),可以任意大。
\begin{xcode}
0
326
-1208
31415926535
\end{xcode}
整数上有我们熟悉的算术运算:
\begin{quote}
\begin{tabular}{ll}
\hs{+} & 计算两个整数的和\\
\hs{-} & 计算两个整数的差\\
\hs{*} & 计算两个整数的积\\
%\hs{$\^$} & 指数计算 \hs{2$\^$3 = 8}\\
\hs{div} & 计算整数除的商 \hs{div 10 3 = 3}\\
\hs{mod} & 计算整数除的余数 \hs{div 10 3 = 1}\\
\end{tabular}
\end{quote}
整数上有下面的关系操作来比较两个整数:
\begin{quote}
\begin{tabular}{ll}
\hs{>} & 大于\\
\hs{>=} & 大于等于\\
\hs{==} & 等于\\
\hs{/=} & 不等于 \\
\hs{<=} & 小于等于\\
\hs{<} & 小于\\
\end{tabular}
\end{quote}

利用上面的基本操作,我们可以定义整数上的函数。比如,我们可以定义以下函数来判断三个整数是否相等。
\begin{code}
equal3 :: Integer -> Integer -> Integer -> Bool
equal3 x y z = (x==y) && (y==z)
\end{code}

\begin{remark}
  尽管\hs{Integer}能表示任意大的整数,但是实现起来会引入一定的开销。
对于很多应用,只要整数足够打就可以了(比如可以用固定的4个字节来表示)。
这时,我们就可以选用\hs{Int}类型。\hs{Integer}和\hs{Int}之间可以利用下面的函数
进行相互转化。
\begin{xcode}
fromInteger :: Integer -> Int
toInteger   :: Int -> Integer
\end{xcode}
\end{remark}

\begin{exercise}
定义一个函数\hs{different3}来判断三个整数是否都不相等。
\end{exercise}

在定义函数时,我们常常需要根据条件判断选择不同的计算。Haskell的最简单的方法是用条件表达式
\begin{xcode}
if condition then m else n
\end{xcode}
表示如果\hs{condition}是\hs{True}就返回\hs{m},否则返回\hs{n}。比如,
\begin{code}
max :: Integer -> Integer -> Integer
max x y = if x >= y then x else y
\end{code}
这里定义了函数\hs{max}来选择两个输入中大的一个数。如果分歧条件比较多,或者条件比较复杂时,
我们可以选用guard形式来更清晰地表述条件计算。
\begin{code}
max :: Integer -> Integer -> Integer
max x y
  | x >= y    = x
  | otherwise = y
\end{code}
这个定义中有两个guard,从上往下执行。如果第一个guard \hs{x>=y} 成立,它就返回\hs{x},否则,
它就是下一个guard。\hs{otherwise}是一个特殊guard,它总是成立。

\begin{exercise}
定义一个函数返回三个整数的最大值。
\end{exercise}

\section{字符类型:\hs{Char}}

字符类型包括所有字符,而字符使用单引号围绕着,如\hs{'a'}在Haskell中表示字符\hs{a},
\hs{'3'}表示数字字符\hs{3}。还有一些特殊的字符:
\begin{quote}
\begin{tabular}{ll}
\hs{'$\backslash$t'} & tab \\
\hs{'$\backslash$n'} & 换行 \\
\hs{'$\backslash$'} & 反划线 \\
\hs{'$\backslash$''} & 单引号\\
\hs{'$\backslash$"'} & 双引号
\end{tabular}
\end{quote}
有一种标准的编码方式叫做ASCII编码,将字符对应于一个整数。下面两个函数表示了字符与
整数之间的变换。
\begin{xcode}
fromEnum :: Char -> Int
toEnum   :: Int -> Char
\end{xcode}

在ASCII编码中,大写字符,小写字符,数字都是连续从下到大排列的。利用这个性质,我们可以
对字符进行变换。设想我们希望定义一个函数将字符变成大写。为此,我们首先计算大写字母和小写字母之间的间隔:
\begin{code}
offset :: Int
offset = fromEnum 'A' - fromEnum 'a'
\end{code}
然后我们就可以定义这个函数:
\begin{code}
toUpper :: Char -> char
toUpper c = toEnum (fromEnum c + offset)
\end{code}

字符按照他们的编码是可以比较的,因此,我们可以定义下面的函数来判断一个字符是不是数字。
\begin{code}
isDigit :: Char -> Bool
isDigit c = ('0' <= c) && (c <= '9')
\end{code}

\begin{remark}
\hs{toUpper}和\hs{isDigit}都是标准库\hs{Data.Char}的标准函数。
\end{remark}

\begin{exercise}
定义一个将小写字符改为大写字符,但是其他字符保持不变的函数。
\end{exercise}

\begin{exercise}
 定义一个将数字字符变为相应数字的函数。
\begin{code}
char2Int :: Char -> Int
\end{code}
\end{exercise}

\section{字符串类型:\hs{String}}

字符串类型\hs{String}是有字符的序列构成。每个字符串用双引号括起来,如\hs{"This is a string!"}。
字符串上面重要的一个操作是字符串链接\hs{++},如,\hs{"string" ++ " join"} = \hs{"strinng join"}。
由于字符串是字符的序列,我们将来可以用序列上的函数来定义字符串上的新函数。

对于字符串,有两个很有用的函数。一个是\hs{read},将一个字符串变为你想要的类型的值,另一个是\hs{show},
将一个类型的值变为字符串。
\begin{code}
read :: String -> a
show :: a -> String
\end{code}
例如
\begin{xcode}
read "True" = True
read "3"    = 3
show True   = "True"
show 3      = "3"
\end{xcode}

\begin{exercise}
定义函数
\begin{xcode}
onTwoLines :: String -> String -> string
\end{xcode}
使得的它能接受两个字符串,返回一个字符串。这个字符串打印时能将两个字符串上下排列。
\end{exercise}

\begin{exercise}
定义一个函数\hs{romanDigit}将一个数字字符用罗马数字表示的程序,如\hs{romanDigit '7' = "VII"}。
\end{exercise}

\section{浮点数类型:\hs{Float}}

浮点数类型\hs{Float}包含所有的浮点小数,如
\begin{xcode}
0.326
-12.08
31415.926
\end{xcode}
Haskell库函数包括很多浮点数上的函数,如平方根,指数函数,对数函数,三角函数等。同时,整数可以通过\hs{fromInt}变为浮点数,浮点数也可以通过\hs{cerling},\hs{fllor}和\hs{round}变成整数。
下面是一些执行的例子。

\begin{demo}
Prelude> sin (pi/2) + sqrt 2
2.414213562373095
Prelude> floor 5.6
5
Prelude> ceiling 5.6
6
Prelude> round 5.6
6
\end{demo}

\begin{exercise}
给定三条边的长度为\hs{a}, \hs{b}, \hs{c},定义一个函数,判定这三条边能否组成一个三角形。
\end{exercise}

\begin{exercise}
 假设$a$, $b$, $c$为二次方程$a x^2 + bx + c = 0$的系数。
 定义一个函数,求出这个方程解的个数。
\end{exercise}

\chapter{函数式程序设计与开发}

\section{函数式程序设计}

在写函数式程序之前,我们必须进行设计。

\subsection*{理解我们需要做什么}

设计的第一步是确认我们是否理解了我们需要做什么。通常,需要我们解决的问题的描述是非形式化的,
描述不完全或可能无解。比如说,我们需要返回三个数字的中间数。显然,如果给出的三个数是$2$, $4$, $3$,其结果应该为$3$,但是如果三个数是$2$, $4$, $2$,那么结果应该是什么呢?你也可能说是$2$,因为如果我们将这三个数排序,我们会得到$2, 2, 4$,但是你也可以说无解,因为这三个数的最大值是$4$, 最小值是$2$, 而没有这两者之间的数。

在下面的讨论中,我们将考虑第一种选择。

\subsection*{清楚它的类型}

设计函数时,我们应该清楚它的类型。对于上面的例子,我们可以写出下面的类型。
\begin{code}
middleNumber :: Integer -> Integer -> Integer -> Integer
\end{code}
很显然,我们在给出定义时,如果它不满足上面的类型,他一定不会是我们想要定义的函数。

\subsection*{我们已经有哪些可利用的信息}

往往我们不需要从头开发一个程序。比如我们已经有了函数\hs{bigger}和\hs{smaller}能求出两个数字的
最大值和最小值。这时,我们就可以轻松地定义上面的函数:
\begin{code}
middleNumber x y z = smaller (bigger x y) z
\end{code}

\subsection*{将问题分割为简单的问题}

如果我们一下子无法解决问题,我们可以试探着将它分解为简单的问题。然后从解决简单的问题着手。
这时,我们可以问自己,假如我已有我想要的那个函数,那么我怎样解决问题。
对于\hs{middleNumber},我们可以说,如果我们有个韩式\hs{between x y z}能判断\hs{y}是中间数字,那我们可以写出下面的程序。
\begin{code}
middleNumber x y z
   | between y x z = x
   | between x y z = y
   | otherwise     = z
\end{code}

\begin{exercise}
给出\hs{between}的函数定义。
\end{exercise}

\chapter{组合类型}

上一章我们介绍了基本类型及上面的操作。本章介绍两种构造组合类型的方法,组类型(tuple)和列表类型(list)。
组类型和列表类型都能将一些数据组合在一起,但是有所不同。一个组是组合一定个数的确定类型的元素,而列表是组合
不定个数的同一个类型的元素。

为了看清楚差异,我们看一个超市模型。一个超市有很多物品,包括名字和它的价格,如
\begin{xcode}
("Salt: 1kg", 10)
("Crisps", 5)
\end{xcode}
这些物品具有组类型\hs{(String, Int)}:我们用\hs{String}来表示名字,用\hs{Int}来表示价格。
现在一个超市包括很多物品,我们可以用一个列表类型\hs{[(String, Int)]}来表示。每一个元素表示一个物品,如
\begin{code}
[("Salt: 1kg", 10), ("Crisps", 5), ("Sugar", 5)]
\end{code}

\section{组类型}

一个组类型是由很多简单的类型组合而成。一个组类型
\[
(t_1, t_2, \ldots, t_n)
\]
包含下面的值
\[
(v_1,v_2, \ldots, v_n)
\]
其中,$v_1 :: t_1, v_2 :: t_2, \ldots, v_n :: t_n$.
为了增加可读性,我们可以给组类型一个名字,例如,对于超市的物品类型,我们可以定义为:
\begin{code}
type ShopItem = (String, Int)
\end{code}

%\begin{example}
我们可以用组类型来定义一个函数返回一组值。比如下面的函数返回两个整数的最小值和最大值。
\begin{code}
minAndMax :: Integer -> Integer -> (Integer, Integer)
minAndMax x y
  | x >= y     = (y,x)
  | othersise  = (x,y)
\end{code}
%\end{example}

给定一个组类型的值,我们可以通过模式匹配取出其中的元素。比如,我们定义一个函数,
将一个二元组的两个整数相加。
\begin{code}
addPair :: (Integer, Integer) -> Integer
addPair (x,y) = x + y
\end{code}

Haskell里面提供了两个已定义的投影函数来取出二元组的元素。
\begin{code}
fst (x,y) = x
snd (x,y) = y
\end{code}

下面一个例子是用二元组来计算斐波那契数列:
\[
0, 1, 1, 2, 3, 5, \ldots, u, v, u+v, \ldots
\]
除了开始的两个数字以外,后面的任何一个数都是他的前面的两个数的和。
我们首先定义一个函数,给定$n$, 它计算出两个连续的斐波那契数的二元组。
\begin{code}
fib2 n = (fib n, fib (n+1))
\end{code}
给我一个连续的斐波那契数二元组\hs{u,v},我们可以通过下面的函数
计算出下一个二元组:
\begin{code}
fibStep (u,v) = (v, u+v)
\end{code}
因此,我们可以定义\hs{fib2}如下:
\begin{code}
fib2 :: Integer -> (Integer, Integer)
fib2 n
  | n==0      = (0,1)
  | otherwise = fibStep (fib2 (n-1))
\end{code}
利用\hs{fib2},我们可以定义\hs{fib}如下:
\begin{code}
fib2 = fst . fib2
\end{code}
这里我们用了函数合成"."将两个函数合成到一起。

\begin{exercise}
定义一个函数计算某个直线与$x$轴的交叉点。
\end{exercise}

\section{代数类型}

代数类型给用户提供了一个定义新类型的方法。下面我们通过例子来说明代数类型的定义方法。

\subsection{枚举类型}

枚举类型顾名思义就是将类型中的元素通过枚举的方式来定义。比如,拳头/剪子/布的游戏的出拳的
类型可以定义为:
\begin{code}
data Move = Rock | Scissors | Paper
\end{code}
在枚举类型上的函数可以通过模式匹配,定义如何操作每一个元素。下面的函数定义了
两个不同出拳的分数:
\begin{code}
score :: Move -> Move -> Integer
score Rock Rock         = 0
score Rock Paper        = -1
score Rock Scissors     = 1
score Paper Rock        = 1
score Paper Paper       = 0
score Paper Scissors    = -1
score Scissors Rock     = -1
score Scissors Paper    = 1
score Scissors Scissors = 0
\end{code}

\begin{remark}
我们前面介绍的布尔类型内部就是通过枚举类型来定义的。
\begin{code}
data Bool = True | False
\end{code}
\end{remark}

\subsection{积类型}

想组类型一样,积类型描述将几个不同类型的元素组合起来形成一个新的类型。
比如,下面定义了一个人的类型。
\begin{code}
data People = Person Name Age
\end{code}
其中\hs{Name}和\hs{Age}分别是\hs{String}和\hs{int}的代名词:
\begin{code}
type Name = String
type Age = Int
\end{code}
这里\hs{People}是类型名,\hs{Person}是数据构造子。构造子\hs{Person}可以理解为一个函数
\begin{code}
Person :: Name -> Age -> People
\end{code}
它接受一个字符串的名字和一个整数的年龄来构造一个人的信息。例如,下面是\hs{People}类型中的一些元素:
\begin{code}
Person "Mary" 28
Person "Bob"  12
\end{code}

像枚举类型一样,积类型上的函数也是通过模式匹配的方式来定义的。
\begin{code}
showPerson :: People -> String
showPerson (Person n a) = "Name: " ++ show n ++ " Age: " ++ show a
\end{code}

\subsection{和类型}
和类型是为了表示不同的选择的类型。比如,一个形状类型可能是圆形,有个半径,或是一个矩形,有高和宽。
这可以定义为:
\begin{code}
data Shape = Circle Float
           | Rectangle Float Float
\end{code}
这个类型可以包括以下的一些值:
\begin{code}
Circle 3.0
Rectangle 10.0 20.0
\end{code}
在这样的类型上的函数也是通过模式匹配自然地定义。
\begin{code}
isRound :: Shape -> Bool
isRound (Circle _)      = True
isRound (Rectangle _ _) = False
\end{code}
\begin{code}
area :: Shape -> Float
area (Circle r)      = pi * r * r
area (Rectangle h w) = h * w
\end{code}

\begin{exercise}
定义一个函数,求一个形状(\hs{Shape})的周长。
\end{exercise}

\begin{exercise}
将\hs{Shape}类型扩充,增加一个三角形,并定义对应的\hs{isRound}和\hs{area}函数。
\end{exercise}

\begin{exercise}
将\hs{Shape}类型扩充为\hs{NewShape},使得圆形包括圆心,长方形包括对角顶点坐标。定义
一个函数
\begin{code}
move :: Float -> Float -> NewShape -> newShape
\end{code}
使得\hs{move x y s}将形状\hs{s}在横向平移\hs{x}在纵向平移\hs{y}。
\end{exercise}

\begin{exercise}
定义一个函数判定两个\hs{NewShape}的形状是否重叠。
\end{exercise}

\section{列表}

列表是一个函数式程序审计中非常重要的类型,就像数学里集合的重要性一样。一个列表表示
同类型元素的序列。对于每个类型\hs{t},列表类型\hs{[t]}表示元素类型为\hs{t}的序列。
\begin{code}
[1,2,3,4] :: [Integer]
[True, False] :: [Bool]
['a','b','b'] :: [Char]
[(+), (-), (*)] :: [Integer -> Integer]
[[1,2],[2,3],[]] :: [[Integer]]
\end{code}
字符串\hs{String}是字符的列表\hs{[Char]}。

\begin{exercise}
说明集合与序列的区别。
\end{exercise}

列表上有一些非常有用的略写。
\begin{itemize}
  \item \hs{[m..n]}表示\hs{[m,m+1,...,n]}。如果\hs{m}超过\hs{n}则返回空列表。
  \begin{code}
    [1..5] = [1,2,3,4,5]
    [3.1 .. 7.0] = [3.1,4.1,5.1,6.1,7.1]
    ['a'..'e'] = "abcde"
  \end{code}
  \item \hs{[m, p..n]}表示\hs{[m,p,p+s,...,p+ks]},其中\hs{p+ks <= n < p+(k+1)s}。
  \begin{code}
    [7,6..2] = [7,6,5,4,3,2]
    [0.0,0.3 .. 1.0] = [0.0,0.3,0.6,0.8999999999999999]
    ['a','c'..'n'] = "acegikm"
  \end{code}
\end{itemize}

对于集合我们有个非常有用的记法叫做集合闭包:
\begin{code}
{ x | x <- {1,2,3,4}, isEven x}
\end{code}
 表示从集合\hs{{1,2,3,4}}中取出偶数而得到集合\hs{[2,4]}。
对于列表,我们有类似的记法,叫作列表闭包(list comprehension)。
\begin{code}
[ x | x <- [1,2,3,4], isEven x ]
\end{code}
返回列表\hs{[2,4]}。

下面给出一些例子,说明列表闭包非常强大的表现能力。
\begin{code}
[ 2*n | n <- [2,4,7]]
[ isEven n | n <- [2,4,7]]
[ 2*n | n<- [2,4,7], isEven n, n>3]
\end{code}
