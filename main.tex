\documentclass[cn,11pt]{elegantbook}

%\usepackage{minted}

\usepackage{fancyvrb}
\renewcommand{\FancyVerbFormatLine}[1]{\qquad #1}
\DefineVerbatimEnvironment{code}{Verbatim}{fontsize=\small}
\DefineVerbatimEnvironment{xcode}{Verbatim}{fontsize=\small}
\DefineVerbatimEnvironment{example}{Verbatim}{fontsize=\small}
\newcommand{\ignore}[1]{}
\newcommand{\hs}[1]{\ {\tt #1}\ }

\title{函数式程序设计}
\subtitle{尽享程序设计之奥秘:描述what,推理why,演算how!}

\author{胡振江}
\institute{北京大学}
\date{2020年7月20日}
\version{0.0.1}
\bioinfo{电子邮件}{huzj@pku.edu.cn}

\extrainfo{你所温柔正确的人总是难以生存,因为这世界既不温柔,也不正确。—— 比企谷八幡}
\setcounter{tocdepth}{3}
\newcommand{\dollar}{\mbox{\textdollar}}
\lstset{
  mathescape = false}
\logo{logo-blue.png}
\cover{cover.jpg}

% 本文档命令
\usepackage{array}
\newcommand{\ccr}[1]{\makecell{{\color{#1}\rule{1cm}{1cm}}}}

\begin{document}

\maketitle
\frontmatter

\tableofcontents
%\listofchanges

\mainmatter

\part{函数式语言的基础}
\chapter{基本概念}
\label{ch:fp_basicConcept}
\chapter{基本概念}
\label{ch:fp_basicConcept}
\chapter{基本概念}
\label{ch:fp_basicConcept}
\input{fp_basicConcept.lhs}



\chapter{基本数据类型}
\label{sec:fp_basicDataType}
\chapter{基本数据类型}
\label{sec:fp_basicDataType}
\chapter{基本数据类型}
\label{sec:fp_basicDataType}
\input{fp_basicDataType.lhs}



\chapter{递归数据类型}

\chapter{规约模型及程序效率}

\chapter{抽象数据类型}

\chapter{副作用的抽象}

\chapter{应用1: 并行计算}

\chapter{应用2: 数据分析}


\part{函数式程序的演算基础}
\input{calc_basicConcept}
\chapter{序列类型上的程序演算理论}

\chapter{递归函数的结构化}

\chapter{演算规则及其开发}

\chapter{应用1: 最优化问题}

\chapter{应用2: 并行自动化}


\part{函数式算法的设计}
\chapter{基本概念}

\chapter{最小自由数问题}

\chapter{数独游戏}


%第一章:基本概念
%第二章:最小自由数问题
%第三章:最矮树构造问题
%第四章:数独游戏
%第五章:寻找名人
%第六章:贪婪算法的秘密

\part{多范式程序语言中的函数式思维}
\input{mul_intro}
\input{mul_ocaml}
\input{mul_scala}
\input{mul_javascript}
\input{mul_python}

%\input{sample}
\nocite{*}
\printbibliography
\appendix

\end{document}
